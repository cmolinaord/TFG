\documentclass[12pt,a4paper,oneside,titlepage]{report}
\usepackage[utf8]{inputenc}
\usepackage[spanish]{babel}
\usepackage{amsmath}
\usepackage{amsfonts}
\usepackage{amssymb}
\usepackage{makeidx}
\author{Carlos Molina Ordóñez}
\date{20 de Marzo de 2017}
\title{Layout de un canal de lectura multi-paralelizado para un sensor de imagen CMOS}


\begin{document}

\maketitle
\tableofcontents

\chapter{Introducción}

\paragraph{}
En este trabajo se va a estudiar todo el proceso de diseño de un bloque fundamental
en cualquier sensor de imagen CMOS, el canal de lectura, que es el encargado
de convertir la información física recibida (número de fotones captados por los
píxeles), a un número digital para posteriormente ser procesado y almacenado digitalmente.

\paragraph{}
El estudio se va a centrar principalmente en el layout de un canal
lectura y en todos los aspectos a tener en cuenta a la hora de abordar esta tarea.
El layout de un sistema microelectrónico consiste en su implementación física
sobre una oblea de algún material semiconductor, habitualmente silicio. El
diseño de layout está sujeto a una serie de normas y problemas que iremos abordando
con mayor detenimiento a lo largo de la exposición.

\paragraph{}
Para introducir al lector en la materia será necesario describir, aunque sea
brevemente, conceptos sobre sensores de imágen, tecnología CMOS y explicar de manera
sencilla la arquitectura de un canal de lectura habitual.

\paragraph{}
Posteriormente se pasará a analizar en detalle los problemas y cuestiones que se
plantean a la hora de diseñar el layout de bloques analógicos en general,
centrandonos en última instancia en los que afectan directamente a un canal de lectura.

\section{seccion 1}

\subsection{subseccion 1}

\paragraph{Esto es el primer párrafo}

Primera version del documento para probar alguna cosa


\paragraph*{Esto es el segundo parrafo, con asterisco}

\end{document}
