\chapter{Diseño electrónico de un canal de lectura}\label{cap:ro_sch}

\paragraph{}
El canal de lectura es el circuito que lee la información almacenada en los píxeles
tras el tiempo de exposición.
En la arquitectura más habitual, y la que vamos a tratar en este estudio, los
píxeles se leen por columnas, entendiéndose que el chip tiene una dirección
\textit{vertical} y otra \textit{horizontal}. Siguiendo ese criterio, el canal de lectura se
sitúa debajo del array de píxeles de forma que cada columna del canal quede alineada
con cada columna del canal.

\paragraph{}
Idealmente podríamos un canal por columna, y meter en ese espacio todo el circuito
para la columna. Pero, en algunos casos, debido al pequeño tamaño de los píxeles
--- unos 5, 8 ó 10\(\mu m\) --- resulta complicado o incluso imposible diseñar el
layout de dicha circuitería en ese reducido espacio horizontal. De ahora en adelante
usaremos la palabra inglesa ''\textit{pitch}'' para referirnos al espaciado con el
que se repite una estructura periódica como el canal de lectura o el array.

\paragraph{}
Por esta razón, es una práctica común usar el \textit{pitch} de varios píxeles, por
ejemplo 2 ó 4, y así tener más espacio para diseñar el layout del circuito. Por contra,
debemos apilar estos 2 ó 4 canales en filas, lo que veremos que nos trae algunos
problemas a la hora de diseñar el layout.

\section{Estructura general}\label{cap:ro_sch_estructura}

\paragraph{}
El canal de lectura es un circuito analógico que va a convertir el voltage
dado por el pixel tras haber sido expuesto a la luz durante un tiempo y
convertirlo en un valor analógico bien definido entre unos límites que van a
significar \textit{blanco} y \textit{negro}, con una resolución definida.

\paragraph{}
Por tanto, el canal de lectura es principalmente un ADC (en inglés, \textbf{A}nalog
to \textbf{D}igital \textbf{C}onverter, o conversor analógico-digital).
Una arquitectura usada habitualmente es el convertidor de rampa.
%WARNING
%Ver ventajas del convertidor  de rampa
Éste, para la conversión usa un generador de una rampa de voltage que se usa para
comparar contra el valor dado por el píxel. Ćuando ambos valores coinciden,
la salida del comparador cambia de estado mediante un flanco de subida o bajada.
De ésta forma la conversión de un valor de tensión se traduce en la detección temporal
de un flanco.

\paragraph{}
De manera simultánea a la rampa analógica se lanza una rampa digital que va contando
valores desde 0 hasta un número que viene determinado por la resolución, y que va a
evolcionar a la misma velocidad que la analógica. Posteriormente, un circuito digital
detectará el flanco y parará el reloj de la cuenta digital, obteniéndose de esta forma
un valor digital para el valor analógico leído.

\section{Operación de lectura}\label{cap:ro_sch_operacion}

\paragraph{}
Para abordar el diseño de un canal de lectura debemos entender como se realiza la
operación de lectura de los píxeles.
%WARNING
%Meter el equematico de un pixel 5T estandar

%WARNING
%Meter el diagrama de tiempo de una operación de lectura



\section{Arquitectura del comparador de rampa}

\section{ADC}



\section{Rampa analógica}

\section{Fuente de corriente}

\section{Bloques de polarización}

\section{Rampa digital y serialización}
