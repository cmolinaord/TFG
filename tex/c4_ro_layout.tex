\chapter{Diseño del layout de un canal de lectura}

\section{Introducción}

\paragraph{}
El diseño del layout de un canal de lectura en el contexto de un proyecto suele
ser una tarea que ocupa gran parte del periodo de diseño, y además es recomendable
abordar en las primeras fases del proyecto, ya que como hemos comentado, el
layout va a influir, en mayor medida que otros bloques, en el diseño del mismo.
Por tanto, el diseño del canal de lectura suele ser un proceso iterativo de
diseño electrónico, layout, extracción, simulación y eventual rediseño.

\section{Jerarquización}
\paragraph{}
A la hora de estructurar el diseño de layout de cualquier bloque repetitivo, como
es el canal de lectura, es básico idear una estrategia en cuanto a la jerarquización
del mismo, ya que usualmente no se trata de un simple array de instancias iguales
situadas una al lado de la otra.

\paragraph{}
Por otra parte, existen estructuras que pueden requerir diferentes periodicidades.
Y también hay que adaptarse a la resolución del array, en este caso, al número de
columnas.


\section{Aproximación al layout de una columna del canal}

\section{Problemas de layout que afectan al diseño del canal de lectura}

\subsection{Acoplos}

\subsection{Concentración de routing}

\subsection{Consumo}

\subsection{Distribución de señales horizontalmente}

\subsection{Condensadores del ADC}

\section{Bloques adicionales}

\subsection{Redundancia}

\subsection{Binning}

\subsection{Implementación de ROIs}

\section{Aproximación a arquitecturas de canales de lectura más complejas}

\subsection{Aumento de la multiparalelización}

\subsection{Sistemas de bajo ruido}

\subsection{Arquitecturas innovadoras}
Modo corriente
