\chapter{Diseño del layout de un canal de lectura}

\section{Introducción}

\paragraph{}
El diseño del layout de un canal de lectura en el contexto de un proyecto suele
ser una tarea que ocupa gran parte del periodo de diseño, y además es recomendable
abordar en las primeras fases del proyecto, ya que como hemos comentado, el
layout va a influir, en mayor medida que otros bloques, en el diseño del mismo.
Por tanto, el diseño del canal de lectura suele ser un proceso iterativo de
diseño electrónico, layout, extracción, simulación y eventual rediseño.

\section{Jerarquización}
\paragraph{}
A la hora de estructurar el diseño de layout de cualquier bloque repetitivo, como
es el canal de lectura, es básico idear una estrategia en cuanto a la jerarquización
del mismo, ya que usualmente no se trata de un simple array de instancias iguales
situadas una al lado de la otra.

\paragraph{}
Por otra parte, existen estructuras que pueden requerir diferentes periodicidades.
Y también hay que adaptarse a la resolución del array, en este caso, al número de
columnas. Cómo ya hemos visto a lo largo de la descripción del esqumático del canal
nuestra unidad básica de un canal de lectura, a la que nos referiremos como
$ro\_chx1$, ocupará una anchura de 2 columnas de píxeles, es decir, una anchura
de $10\mum$. En esa anchura tiene que ser posible incluir todos los bloques que
se repiten por canal, es decir, la fuente de corriente y el ADC de una columna de
píxeles.

\paragraph{}
En un siguiente nivel de jerarquía, incluiremos 8 canales unitarios y sus
correspondientes bloques de polarización (PXCS\_BIAS y ADC\_BIAS), que cómo vimos
están compartidos por cada 8 canales. De ésta forma, el bloque $ro\_chx8$
tendría una anchura de $10\mum \times 8 = 80\mum$, y tendríamos canales para
leer 16 columnas de píxeles.

\paragraph{}
Desde 16 columnas hasta las columnas totales del array podemos elegir varios niveles
de jerarquías intermedios. Considerando que nuestro array tiene la estructura
mostrada en el capítulo \ref{cap:pxa_array}, en la figura \ref{fig:pxa_array},
con 2560 columnas activas y 64 columnas oscuras, vamos a hacer una jerarquía que
cubra esos 64 píxeles, es decir, 32 canales, por lo que se llamará $ro\_chx32$ e
instanciará 4 unidades de $ro\_chx8$.

\paragraph{}
Para completar las 2560 columnas de píxeles activos, sólo hay que instanciar 40
de los bloques de 32 canales. Y para completar el array, tan sólo hay que añadir
2 canales (4 píxeles) a izquierda y derecha de la zona activa (píxeles dummy), y
un bloque de 32 canales en la zona de las columnas oscuras, con sendos pares de
canales a izquierda y derecha para hacer las columnas oscuras dummy.

\subsection{Redundancia}

\paragraph{}
En todo diseño microelectrónico, debido a defectos en la fabricación, ciertos
dispositivos pueden resultar completamente inutilizables o, al menos, con prestaciones
por debajo de lo esperado en relación al resto de dispositivos equivalentes.
Ésto se acentua en casos como un canal de lectura, en los cuales la integración
y la congestión de rutado y dispositivos es muy alta, y además tiene un patrón
altamente simétrico y periódico.

\paragraph{}
Con el paso del tiempo, los fabricantes de sensores de imágen han detectado que
el canal de lectura sufre de considerablemente más fallos que el resto de circuitos
en el sensor, y por consiguiente es uno de los principales contribuyentes a la
disminución del "\textit{yield}" (del inglés: rendimiento), que da cuenta del
porcentaje de sensores por oblea que no funcionan o son defectuosos, lo que va
en contra del beneficio económico.

\paragraph{}
Para tratar de disminuir éste problema, se suele poner en práctica el uso de canales
redundantes, ésto es, canales adicionales que podrían ser usados en el caso de que,
una vez fabricado el chip, se detecten fallos en un canal concreto.

\paragraph{}
Para ésto se añade un canal adicional cada cierto número de canales y mediante un
sistema de llaves a la entrada de todos los canales podemos redireccionar la
salida todos los píxeles en el rango afectado y posteriores al canal del fallo,
hacia el siguiente canal. Con ésto se puede ver fácilmente que podemos solucionar
tantos fallos como canales redundantes sean incluidos, siempre y cuando no se
produzca más de un fallo por región.

\begin{figure}[h]
	\includesvg[width=\textwidth]{svg/redundancy.svg}
	\caption{Ejemplo del funcionamiento del sistema de redundancia}
	\label{fig:redundancy}
\end{figure}

\paragraph{}
En la figura \ref{fig:redundancy} vemos un ejemplo de aplicación del sistema de
redundancia dónde tendríamos un canal redundante por cada 4. En el que caso de que
fallen los canales 2 y 5 mostrados en la imagen, el bloque de redundancia colocado
en la parte superior del canal de lectura podría dirigir cada columna a la siguiente,
de la forma que muestran las líneas rojas.

\paragraph{}
Cómo ya se puede intuir, la anchura del canal de lectura, en el supuesto de usar
redundancia se tiene que ver reducida con objeto de incluir un canal adicional
cada cierto tiempo. En nuestro caso vamos a incluir un canal redundante por cada
grupo de 32 canales funcionando. Con ésto, haciendo una sencilla cuenta, podemos
establecer la nueva anchura en $9\mum$ en vez de las $10\mum$ originales. Ésto nos
permite además tener una anchura sobrante cada 33 canales que será:
$32\cdot 10\mum - 33\cdot 23\mum = 23\mum$ de anchura sobrante.

%WARNING
%Continuar aqui

\section{Aproximación al layout de una columna del canal}

\section{Problemas de layout que afectan al diseño del canal de lectura}

\subsection{Acoplos}

\subsection{Concentración de routing}

\subsection{Consumo}

\subsection{Distribución de señales horizontalmente}

\subsection{Condensadores del ADC}

\section{Bloques adicionales}


\subsection{Binning}

\subsection{Implementación de ROIs}

\section{Aproximación a arquitecturas de canales de lectura más complejas}

\subsection{Aumento de la multiparalelización}

\subsection{Sistemas de bajo ruido}

\subsection{Arquitecturas innovadoras}
Modo corriente
