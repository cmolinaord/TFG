\chapter*{Resumen}
\addcontentsline{toc}{chapter}{Resumen}\label{cap:resumen}

\paragraph{}
En éste trabajo se va a presentar y estudiar el layout de un canal de lectura
para sensores de imagen CMOS y los problemas y aspectos importantes que considerar
en el proceso de diseño y verificaciones, incluyendo los procesos de extración de
parásitos y simulación previa y posterior al layout.

\paragraph{}
El canal de lectura de un sensor de imágen es el bloque analógico encargado de
transformar la señal de voltaje analógico recibida del píxel, en un valor digital.
En la mayoría de los casos, también el que se va a tratar aquí, el canal de lectura
consta de un convertidor de rampa cuyos comparadores se distribuyen uno por
columna, o grupo de columnas, de píxeles, siendo la generación de la rampa, externa
al bloque del canal.

\paragraph{}
El proceso de diseño del canal incluye varias fases desde el estudio inicial, el diseño
del esquemático, hasta el diseño del layout y verificaciones físicas, acabando con
las simulaciones de su funcionamiento. En éste trabajo nos vamos a centrar en la
parte del diseño de layout y su posterior extraído y simulación, que como veremos
es un punto muy importante en contexto del proyecto.
