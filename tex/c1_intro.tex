\chapter{Introdución}

\paragraph{}
En este trabajo se va a estudiar el proceso de diseño de un bloque fundamental
en cualquier sensor de imagen CMOS, el canal de lectura, que es el encargado
de convertir la información física recibida, que en esencia es el número de fotones
captados cada pixel, a una señal electrónica analógica que posteriormente será digitalizada,
procesada y, eventualmente, almacenada.

\paragraph{}
El estudio se va a centrar principalmente en el layout de este canal de
lectura y en todos los aspectos a tener en cuenta a la hora de abordar esta tarea.
El layout de un sistema microelectrónico consiste en su implementación física
sobre una oblea de algún material semiconductor, típicamente silicio cristalino. El
diseño de layout está sujeto a una serie de normas y problemas que iremos tratando
con mayor detenimiento a lo largo de la exposición.

\paragraph{}
Para introducir al lector en la materia será necesario describir, aunque sea
brevemente, conceptos sobre sensores de imágen, tecnología CMOS y explicar de manera
sencilla la arquitectura de un canal de lectura habitual.

\paragraph{}
Posteriormente se pasará a analizar en detalle los problemas y cuestiones que se
plantean a la hora de diseñar el layout de bloques analógicos en general,
centrandonos en última instancia en los que afectan directamente a un canal de lectura.

\section{Sensores de imagen}

\paragraph{}
Un sensor de imagen o cámara fotográfica es, en esencia, un sistema que capta
una imagen instantánea de una escena mediante la luz que emiten los objetos que
se encuentran en su campo de visión y que llegan a una pantalla donde se almacena
la información que proyecta ese rayo de luz, ya sea por un proceso químico
o electrónico, que es el caso que se va a tratar aquí.

\paragraph{}
En cuanto a los sensores de imágenes electrónicos se pueden distinguir dos tipos
principalmente, los CCD (\textit{Charge-Coupled Device}) y los CMOS
(\textit{Complementary Metal-Oxide-Semiconductor}). Las diferencias entre ellos
se basan en la tecnología empleada y en la forma de leer el array de pixeles.

\subsection{Sensores de imagen CCD}

\paragraph{}
{\huge\textbf{Leer algo en el libro de sensores de imagen!!!!!!!!!}}

\subsection{Sensores de imagen CMOS}

\paragraph{}


\section{Arquitectura de sensores de imagen CMOS}

\section{Array de píxeles}

\paragraph{}
El dispositivo principal en un sensor de imagen es el píxel, que es el elemento
receptor de la luz y el encargado de hacer la primera conversión de fotones a
electrones. Estos fotones se traducen en electrones por medio del efecto fotoeléctrico
descrito por Albert Einstein: Cada fotón que incida en la superficie del silicio
es capáz de dar la energía suficiente a un electrón de la banda de valencia para que
pase a la banda de conducción y se pueda mover libremente por la red cristalina.
Si ahora aplicamos un campo eléctrico podemos guiar a todos los electrones
fotogenerados hacia un pozo de potencial donde almacenarlos temporalmente. Este
es el llamado proceso de exposición, que tiene una duración definida. En función
del tiempo de exposición y de la intensidad lumínica recibida por el pixel, este
se cargará con mayor o menor número de electrones.

\paragraph{}
Observando el esquemático de un pixel 5T (5 transistores) habitual, podemos entender
el funcionamiento del mismo. El fotodiodo (PD) representa la zona donde los fotones
inciden y donde se convierten a electrones durante el tiempo de exposición. Al final
de éste periodo se abre la compuerta (TX), también llamada \textit{transfer gate},
y los electrones pasan por difusión hacia un pozo de potencial llamado \textit{floating
diffusion} (FD), que se representa como un condensador.

\paragraph{}
Posteriormente, la carga almacenada en la \textit{floating diffusion} se convierte
en voltaje a través del seguidor por fuente (SF) cuando la fila actual es seleccionada
para ser leída mediante la activación del transistor de SEL, que carga la columna
de lectura con la tensión a la salida del \textit{source follower}. Este valor de
tensión es el que será leído por el Canal de Lectura.

\paragraph{}
Durante todo el proceso
se pueden producir dos operaciones de limpieza o puesta a cero. Una la hace el
transistor de anti-blooming (BMG), que se encarga de limpiar el diodo de posibles
electrones fotogenerados antes del tiempo de exposición. Este transistor también
realiza la fucnión de "desagüe" en el caso de que la intensidad lumínica sea tal
que el número de electrones en el fotodiodo sature, en cuyo caso, este transistor
drenará la carga sobrante hacia la alimentación. La otra limpieza la
lleva a cabo el transistor de Reset (RST), que reestableze el valor "cero" en la
 \textit{floating diffusion} antes de volcar los nuevos electrones generados en
 la exposición actual.




\paragraph{}
El canal de lectura de un sensor CMOS, referido habitualmente por sus
siglas en inglés \textbf{RO} (\textit{Read-Out Channel}), es el bloque que se
encarga de traducir el voltaje almacenado en cada pixel durante el proceso
de exposición, en un número digital. Esta descripción concuerda con el concepto
ampliamente utilizado en electrónica de ADC, siglas en inglés de \textit{Analog-to-Digital
Converter}, (Convertidor Analógico-Digital), que, en general toma una señal
analógica y la expresa en valores discretizados
