\chapter{Tecnología CMOS}

\section{Introducción}

\paragraph{}
La tecnología CMOS, que es ampliamente utilizada en el diseño de circuitos integrados
en la actualidad, se basa en la posibilidad de integrar en un mismo substrato
semiconductores con ambos dopados (N y P). Con ello podemos implementar transistores
MOSFET tanto PMOS como NMOS en un mismo diseño.

\paragraph{}
Es una tecnología que cumple ya los
50 años, debido a que empezó a ponerse en práctica a mediados de los años 1960.
Inicialmente se usó principalmente en circuitos digitales, debido a que los
transistores CMOS solo consumen potencia cuando conmutan, a diferencia de los
transistores de unión bipolar. Por otra parte, es más sencillo disminuir su tamaño
y tienen un menor coste de fabricación.

\paragraph{}
Poco a poco se fue introduciendo la tecnología CMOS en el diseño de circuitos
analógicos. Los bajos costes y la posibilidad de crear circuitos digitales y
analógicos en el mismo chip hacían esta opción muy interesante. Pero aún así, los
transistores bipolares eran mucho menos ruidosos y más rápidos que los MOSFET, por
lo que la trancisión fue lenta. Con el desarrollo de la tecnología CMOS, la velocidad
y el ruido de éstos, se ha visto muy mejorada, y actualmente domina el mercado,
aunque en muchos casos se sigue usando tecnología bipolar.
% Mirar un poco más esto (Historia y actual uso de las tecnologías del silicio)


\section{Proceso de fabricación}

\paragraph{}
La gran mayoría de los circuitos integrados CMOS estan construidos sobre silicio.
El silicio (Si), elemento de número atómico 14, es muy abundante en la Tierra, aunque
no se encuentra de forma pura, sino como óxidos de silicio o silicatos. Entre los
óxidos de silicio, basados fundamentalmente en la sílice o dióxido de silicio
(SiO\textsubscript{2}), se encuentran el cuarzo y el sílex, ambos ampliamente extendidos
en la corteza terrestre. Los silicatos son sales basadas en el ión silicato (SiO\textsubscript{4}),
y forman parte de minereales como los feldespatos, micas, berilio.

{\huge\textbf{Leer algo en el libro de minerales y rocas en casa!!!!!!!!!}}

\paragraph{}
A pesar de estar en tan alta abundancia en la Tierra, como se dijo antes, el silicio
puro no se dá naturalmente, debe ser refinado y cristalizado. El proceso consiste,
tratado de manera sencilla, en extraer el óxigeno de los compuestos mencionados arriba
a base de añadir carbono y fundir la mezcla en un horno. Tras éste y otros porcesos
obtendríamos silicio relativamente puro, pero en forma policristalina (habitualmente
nos referiremos a éste como polisilicio). Esta forma contiene silicio puro, pero
cristalizado en pequeños cristales independientes con diferentes planos cristalinos
creando efectos de borde en el interior del conglomerado, que anulan los efectos
semiconductores del silicio.

\paragraph{}
Para construir un único monocristal de silicio se suele usar el llamado proceso
de Czochralski, en el cual, una varilla de silicio usada como semilla se va
rotando en un baño de silicio puro fundido a unos 1400\grad C de manera que va creciendo
en diámetro a medida que los átomos se silicio se van depositando en la capa externa.

\paragraph{}
Lo que queda es un lingote de aproximadamente un metro de largo y pocas decenas
de centímetro de diámetro de silicio monocristalino siguiendo la esctructura cristalina del
diamante.

\paragraph{}
Para ser usado en la industria de semiconductores, estos lingotes se deben laminar
en obleas de pocos milímetros de espesor sobre las que se implementarán los
circuitos integrados, que se suelen distribuir en una matriz ocupando toda la
superficie de la oblea que luego será cortada para separar cada "dado".

\paragraph{}
Con estos procesos tendríamos tan solo el substrato de silicio sobre el que
se debe construir el circuito, los transistores y otros dispositivos que se
necesiten. Esto se hace mediante fotolitografía, una técnica que crea permite
construir cada capa mediante la projección de una luz ultravioleta a través de
una máscara que permite o no pasar la luz según el layout diseñado.

\paragraph{}
Para empezar se recubre toda la oblea con un material fotosensible a la luz UV,
que dependiendo de si recibe luz o no, cambia sus propiedades haciendo que donde
ha recibido luz pueda ser eliminado posteriormente y quedar sólo donde no se recibió
luz, o viceversa. De esta forma obtenemos un recubrimiento selectivo con este
material, haciendo que, en un paso posterior podamos hacer crecer, sobre las zonas
sin recubrimiento, una capa de óxido de silicio si queremos un aislante, o de polisicio,
cuyos usos se tratarán más adelante, o de metal como condutor, habitualmente
aluminio o cobre, o algún tipo de dopado para crear las difusiones por ejemplo.

\paragraph{}
Tras las creación de cada capa, en la mayoría de los casos se realiza un pulido
fino de la superficie para que la siguiente capa asiente correctamente sobre una
superficie plana.

\section{Transistores CMOS}
\paragraph{}
El dispositivo fundamental que usa en cualquier circuito integrado CMOS es el
transistor CMOS, ya que es la base de funcionamiento de circuitos tanto digitales
(puertas lógicas, inversores, flip-flops, buffers), como de circuitos analógicos
(amplificadores operacionales y de transconductacia, convertidores analógico-digital,
referencias de tensión, bandgap).

\paragraph{}



\section{Técnicas avanzadas en tecnología CMOS}

\section{Diseño de layout}

\paragraph{}
El término \textit{layout} hace referencia a la implementación física del circuito
electrónico que se quiere fabricar. El layout consiste en un dibujo con toda
la información que necesita la empresa fabricante para implementar el circuito
sobre una oblea de silicio. Dicha información se representa por medio de "capas"
que se distribuyen en un espacio bidimensional. Cada capa tiene un significado y
unas normas que cumplir. El trabajo del diseñador de layout es definir, dibujar y
verificar el layout de los diferentes bloques que componen el chip siguiendo
unas normas dadas por el fabricante.

\subsection{Capas de layout}

\paragraph{}
Todo diseño de layout está compuesto de una cierta cantidad de capas, de las cuales
algunas tienen significado físico directo y otras son capas que usa el software de
diseño para verificaciones.

\subsection{Herramientas de CAD}

\subsection{Problemas habituales en el diseño de layout}
