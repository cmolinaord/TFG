\chapter*{Objetivos y metodología}
\addcontentsline{toc}{chapter}{Objetivos y metodología}\label{cap:objetivos}

\paragraph{}
Con este trabajo se pretende hacer una recopilación de los problemas y de los
puntos claves a la hora de diseñar un canal de lectura. Se trata de una guía
que describe los aspectos más críticos cuando se está diseñando el layout de un
canal de lectura, haciendo una introducción en los primeros capítulos tanto a
la tecnología CMOS usada, como a la arquitectura electrónica de un canal de lectura.

\paragraph{}
La exposición de dichos problemas y consejos se hace usando un canal de lectura
de ejemplo, del cual se describen una serie de características concretas a lo
largo del trabajo. Aunque también cabe decir que mucho de lo que se va a tratar
bien podría aplicarse a canales ligeramente diferentes, con, por ejemplo,
diferente número de columnas, pitch o número de canales apilados.

\paragraph{}
En cuanto a la tecnología usada, se ha elegido una de 180nm y de 4 metales que
define la compañía TowerJazz\textsuperscript{\textregistered} especialmente para la fabricación de
sensores de imagen CMOS\cite{TowerJazz}, que provee de dispositivos para los
píxeles como microlentes, array de color CFA (\textit{Color Filter Array}) y \textit{stiching}
(tecnología de pegado de dados del chip cuando el tamaño excede el tamaño máximo
de dado).

\begin{table}[h]
	\centering
	\caption{Especificaciones usadas en el diseño}
	\label{table:specifications}
	\begin{tabular}{l*{7}{c}r}
		\hline
		Especificaciones	& Mín & Valor típico & Máx \\
		\hline
		Tecnología		&  & TS18 & \\
		Número de metales	&  & 4 & \\
		VDDA			& 3.0 & 3.3 V & 3.6 V \\
		VDDD 			& 1.6 & 1.8 V & 2.0 V \\
		Temperatura		& -20\centigrade & 27\centigrade & 90\centigrade \\
		\hline
	\end{tabular}
\end{table}

\paragraph{}
El esquemático del bloque se ha diseñado usando \textit{Virtuoso Schematic Editor}\cite{VirtuosoSchematic},
el layout con \textit{Virtuoso Layout XL}\cite{VirtuosoLayoutXL} y las
simulaciones con su paquete ADE (\textit{Virtuoso Analog Design Environment}), todos
ellos propiedad de la popular marca CADENCE\textsuperscript{\textregistered}.
En cuanto a la extracción de parásitos y las verificaciones físicas de LVS
(\textit{Layout versus Schematic}), DRC (\textit{Design Rules Check}), ERC
(\textit{Electrical Rule Check}) y \textit{Antenna}, se han realizado con el entorno
de CALIBRE\textsuperscript{\textregistered} \cite{Calibre}.
