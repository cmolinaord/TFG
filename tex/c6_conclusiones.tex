\chapter{Conclusiones}

Tras todos los puntos expuestos en el trabajo, podemos concluir que el canal de
lectura es un bloque que requiere de una cuidadosa planificación dentro del
contexto del proyecto de diseño de un sensor de imágen. Los responsables del
proyecto deben dedicar un tiempo amplio al estudio, diseño, layout, simulaciones
y verificaciones del bloque.\\

Los canales de lectura son un tema muy presente en la literatura actual por parte
de grupos de investigación y empresas de todo el mundo. El expuesto aquí es un
canal estándar dónde podemos apreciar bastantes puntos de la dificultad de su
diseño, pero existen muchas vías de investigación abiertas a otras arquitecturas,
nuevas funcionalidades, mayores resoluciones o menor ruido.\\

Por ejemplo existen arquitecturas innovadoras para convertidores de mayor resolución
ideados para poder alcanzar menores niveles de ruido y poder captar imágenes en
condiciones de baja iluminación. Un ejemplo de como conseguir esta mayor resolución
puede ser dividiendo el ADC en dos etapas, como se expone en el artículo de la
bibliografía\cite{Gonzalez-Charlet2015:Low_noise} sobre un sensor de bajo ruido
llamado LoNIS (\textit{Low Noise Image Sensor}), desarrollado en AnaFocus\cite{anafocus}.\\

Con lo expuesto en el trabajo, una de las conclusiones que podemos sacar es que
el layout del canal de lectura puede condicionar ciertos aspectos del diseño del
electrónico del mismo. Por ejemplo, algún acoplo que no podamos reducir porque
imposibilidad en el layout, podría hacernos replantearnos el funcionamiento
del bloque, o imponer una serie de restricciones a la hora de usarlo.\\

Todo esto nos lleva a otro punto de especial importancia, que es la necesidad de
realizar un exhaustivo y detallado estudio de viabilidad. Éste se debe hacer
previo al inicio del proyecto y debe tomar las especificaciones requeridas por el
producto final y realizar los estudios pertinentes para saber si será posible la
realización del diseño en tiempo, sin grandes costes y tratando de prevenir problemas
y/o dificultades durante su puesta en práctica. Y, por supuesto, dicho estudio debe
incluir el layout, y los problemas mencionados durante éste trabajo.\\

Por otra parte, en el ámbito de los sensores de imágen, existen innovaciones en
cuanto a los píxeles. Éste es un tema en el que hay mucho por investigar aún, ya
que en busca de nuevas mejoras se están investigando píxeles muy interesantes para
aplicaciones 3D, usando la técnica de ``ToF'' (\textit{Time of Flight},
tiempo de vuelo)\cite{Portaluppi2018}\cite{Payne2014}.
Éstos píxeles deben ser capaces de captar un pulso de luz emitido desde el sensor
y calcular el tiempo que tarda en recibirse tras haber rebotado en algún objeto,
para calcular la distancia al mismo. Para ésto es necesario, además de píxeles
que puedan transferir su carga mucho más rápido, mejoras en el canal de lectura
que permitan esa integración de la señal recibida.\\
