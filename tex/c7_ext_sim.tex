\chapter{Extracción y simulación}

\paragraph{}
Cómo se ha ido presentando durante todo el trabajo, todo el trabajo de diseño,
tanto esquemático cómo de layout, no tiene sentido sin hablar de un proceso de
verificación de que lo que se ha diseñado funciona cómo queremos que funcione,
y ésto queremos comprobarlo, lógicamente, antes de que se fabrique el chip.

\paragraph{}
Las técnicas de simulación y extraído han evolucionado mucho en los últimos tiempos,
con mejores algoritmos y procesadores más potentes ahora podemos simular circuitos
mucho mayores que años atrás, llegando a los cientos de miles de transistores o
%WARNING
%Ajustar numero de transistores en simulaciones gordas

\paragraph{}
El proceso de diseño es iterativo, se diseña según unos cálculos previos, se hacen
simulaciones, primero a nivel de bloques pequeños y luego bloques más grandes y
en función de los resultados se pueden ajustar algunos parámetros en el esquemático.
Para afinar aún más siempre es necesario realizar también simulaciones post-layout,
esto es, habiendo hecho previamente un extraído de parásitos debidos al layout.

\section{Extracción de parásitos}\label{cap:extraccion}

\paragraph{}
El esquemático que se diseña en los primeros pasos, habitualmente no tiene ninguna
componente debida a aspectos físicos de layout. Las conexiones entre dispositivos
son ideales, sin resistencia ni capacidades ni inductancias, pero éstas existirán
una vez fabricado en el silicio.

\paragraph{}
Quizás en algunos casos como circuitos estáticos, de bajo consumo, o con amplios
márgenes de diseño, el extraído de parásitos no es algo crítico puesto que van a
afectar poco al comportamiento general del circuito, aunque como norma se prefiere
realizar siempre una comprobación. En otros casos en cambio, como puede suceder
en el caso del canal de lectura, ésta extracción es de suma importancia y hay
que dedicarle un tiempo considerable a dicho proceso iterativo, en ocasiones
alargándose más que el propio diseño inicial.

\paragraph{}
Los extraídos, por regla general se pueden hacer sólo capacitivos, sólo resistivos
o ambos, y en ocasiones algunos extractores permiten extraer inductancias. El extraído
capacitivo además se puede hacer en 2 dimensiones, entendiendo con ésto que sólo
se consideran las capacidades inmediatas, entre 2 metales a la misma altura o
en capas consecutivas en altura. En contraposición podemos afinar más con un
extraído 3D que sí considera líneas más alejadas sin acoplo directo. Para nuestro
caso es muy recomendable ésta alternativa dada la alta concentración de routing y
la importancia de los acoplos.

\paragraph{}
El extraído es un proceso costoso computacionalmente, más aún si lo hacemos con alta
resolución, 3D y tratándose de bloques grandes. Lo recomendable es extraer mas de una
sóla columna de canal para ver posibles acoplos entre canales adyacentes y posibles
asimetrías entre ellos debido a la asimetría introducida por el hecho de compartir
bloques de polarización cada 8 canales. Ésto implica que el layout de una columna
de ADC, por ejemplo, a pesar de tener idéntico layout, puede llevar algunas señales
verticales diferentes.

\section{Simulación}\label{cap:simulacion}
