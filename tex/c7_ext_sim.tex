\chapter{Extracción y simulación}

\paragraph{}
Cómo se ha ido presentando durante todo el trabajo, todo el trabajo de diseño,
tanto esquemático cómo de layout, no tiene sentido sin hablar de un proceso de
verificación de que lo que se ha diseñado funciona cómo queremos que funcione,
y ésto queremos comprobarlo, lógicamente, antes de que se fabrique el chip.

\paragraph{}
Las técnicas de simulación y extraído han evolucionado mucho en los últimos tiempos,
con mejores algoritmos y procesadores más potentes ahora podemos simular circuitos
mucho mayores que años atrás, llegando a los cientos de miles de transistores o
%WARNING
%Ajustar numero de transistores en simulaciones gordas

\paragraph{}
El proceso de diseño es iterativo, se diseña según unos cálculos previos, se hacen
simulaciones, primero a nivel de bloques pequeños y luego bloques más grandes y
en función de los resultados se pueden ajustar algunos parámetros en el esquemático.
Para afinar aún más siempre es necesario realizar también simulaciones post-layout,
esto es, habiendo hecho previamente un extraído de parásitos debidos al layout.

\section{Extracción de parásitos}\label{cap:extraccion}

\paragraph{}
El esquemático que se diseña en los primeros pasos, habitualmente no tiene ninguna
componente debida a aspectos físicos de layout. Las conexiones entre dispositivos
son ideales, sin resistencia ni capacidades ni inductancias, pero éstas existirán
una vez fabricado en el silicio.

\paragraph{}
Quizás en algunos casos como circuitos estáticos, de bajo consumo, o con amplios
márgenes de diseño, el extraído de parásitos no es algo crítico puesto que van a
afectar poco al comportamiento general del circuito, aunque como norma se prefiere
realizar siempre una comprobación. En otros casos en cambio, como puede suceder
en el caso del canal de lectura, ésta extracción es de suma importancia y hay
que dedicarle un tiempo considerable a dicho proceso iterativo, en ocasiones
alargándose más que el propio diseño inicial.

\paragraph{}
Los extraídos, por regla general se pueden hacer sólo capacitivos, sólo resistivos
o ambos, y en ocasiones algunos extractores permiten extraer inductancias. El extraído
capacitivo además se puede hacer en 2 dimensiones, entendiendo con ésto que sólo
se consideran las capacidades inmediatas, entre 2 metales a la misma altura o
en capas consecutivas en altura. En contraposición podemos afinar más con un
extraído 3D que sí considera líneas más alejadas sin acoplo directo. Para nuestro
caso es muy recomendable ésta alternativa dada la alta concentración de routing y
la importancia de los acoplos.

\paragraph{}
El extraído es un proceso costoso computacionalmente, más aún si lo hacemos con alta
resolución, 3D y tratándose de bloques grandes. Lo recomendable es extraer mas de una
sóla columna de canal para ver posibles acoplos entre canales adyacentes y posibles
asimetrías entre ellos debido a la asimetría introducida por el hecho de compartir
bloques de polarización cada 8 canales. Ésto implica que el layout de una columna
de ADC, por ejemplo, a pesar de tener idéntico layout, puede llevar algunas señales
verticales diferentes.

\paragraph{}
Lo que una extracción hace básicamente es calcular las capacidades que se ven
entre líneas cercanas por layout y resistencias debidas a líneas metálicas o
de poly. Posteriormente crea un nuevo circuito añadiendo todas las capacidades
y resistencias que sean mayores que un cierto umbral seleccionado.

\section{Simulación}\label{cap:simulacion}

\paragraph{}
La simuación, como ya hemos dicho, es un punto importante en el diseño de un circuito
microelectrónico. Con una simulación se trata de modelar o imitar el comportamiento
de los dispositivos cuando estén fabricados en silicio. Para lo cual, cada
dispositivo instanciado consta de un model que define cómo son las tensiones y
corrientes que se establezerán en sus nodos en función de las tensiones y/o
corrientes que reciba y de las cualidades y dimensiones del dispositivo en concreto.

\paragraph{}
Tras establecer las relaciones de dependencias de unas variables con otras en la malla
del circuito, el simulador resolverá el sistema de ecuaciones, y, si converge, dará
una solución.

\paragraph{}
Las simulaciones pueden hacerse de varios tipos, estáticas o \textit{dc}, éstas
sólo calculan el punto de operación de todos los dispositivos y las corrientes y
voltajes en cada rama y nodo, las dinámicas o transitorias, que calculan las
evoluciones temporales de los voltajes y tensiones de los nodos y ramas partiendo
desde una condiciones iniciales. Adicionalmente hay otros tipos de simulaciones,
cómo barridos en frecuencia (\textit{ac}), análisis de ruido en frecuencia, que
tiene en cuenta los ruidos que generan los distintos dispositivos, análisis temporal
de ruido, que es una simulación transitoria pero con ruidos añadidos.

\paragraph{}
Al igual que en el diseño de cualquier bloque, el diseñador debe pasar por varias
etapas de simulado. Inicialmente, tras el diseño de los bloques pequeños, se harán
simulaciones sencillas para comprobar que el bloque diseñado se ajusta a los valores
calculados de punto de operación, ganancia, respuesta temporal, ruido...

\paragraph{}
En un estado más avanzado de diseño se integrarán los diferentes bloques unitarios
para comprobar que juntos siguen actuando como se espera, sin fijarnos tanto en el
comportamiento específico de una parte del mismo, sino en el rendimiento general del
conjunto.

\paragraph{}
En cuanto el layout de los bloques esté diseñado, al menos una primera versión, es
recomendable empezar a simularlo usando el extraído del mismo, ya que en según que
casos, puede afectar considerablemente al rendimiento general. En una simulación
post-layout lo que esperamos obtener son por ejemplo, formas de ondas más suaves
y al mismo tiempo con pequeñas perturbaciones, al contrario de las simulaciones
con esquemático, que suelen dar formas de ondas con bordes abruptos y sin perturbaciones.

\paragraph{}
Si por alguna circunstancia no es posible simular el circuito con el extraído, ya
sea porque todavía no está diseñado, o porque es demasiado grande y es difícil
la extracción, siempre es buena práctica realizar alguna simulación en la que el
diseñador incluya ``a mano'' los parásitos que considere más importante,
calculándolos del existente o previsto layout.

\paragraph{}
Ésta circunstancia se da muy habitualmente en el caso de las alimentaciones por ejemplo.
Normalmente no es viable hacer un extraído de toda la línea que alimenta a un circuito
desde los PADs de entrada al chip. Y por otro lado, sólo nos interesa la posible
caída de tensión que se produzca por el camino, y ésta se obtiene con un cálculo
simple. Añadiremos una resistencia estimada usando la resistividad, anchura y longitud
de cada alimentación. Además se suele incluir un modelo experimental del PAD de entrada.
El PAD es el área metálica cuyo aislante superior se eliminará para poder acceder
con un hilo de oro y conectar una señal o alimentación interna con el exterior.
El PAD consta normalmente de algunos diodos de protección ESD (descarga electrostática).
Lo que se suele hacer habitualmente es simular usando el extraído de dicho PAD e
incluir además un modelo eléctrico del hilo de oro, mediante una resistencia,
capacidad e inductancia determinadas experimentalmente. La inclusión de dicho modelo
podrá causar que en nuestra simulación veamos caídas de tensiones o efectos de
segundo órden como rebotes o rizados en las señales, que se deberían estudiar y
tratar de reducir.

\paragraph{}
Otro caso en el que no podríamos usar el extraído de todo el bloque sería al tratar
de simular todo el canal de lectura, en nuestro caso, las 1320 columnas de canales
correspondientes a las 2640 colúmnas de píxeles totales, activos, OB y dummies.
Una simulación de el canal al completo es interesante desde el punto de vista de
sistemas para analizar si hay problemas debidos a la propagación como el retraso
o la degradación de las señales. Pero lo cierto es que no estamos interesados
en obtener datos perfectos y precisos de todas y cada una de las columnas, sino que
habitualmente sólo queremos conocer el comportamiento en algunas columnas concretas,
como por ejemplo la primera, última y alguna de la parte central.

%WARNING
%Continuar aqui, explicar como es la simulacion de Alex
