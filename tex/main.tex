\documentclass[12pt,a4paper,oneside,titlepage]{report}
\usepackage[utf8]{inputenc}
\usepackage[spanish]{babel}
\usepackage{amsmath}
\usepackage{amsfonts}
\usepackage{amssymb}
\usepackage{makeidx}
\usepackage[top=2.5cm, bottom=2.5cm, left=2.5cm, right=2.5cm]{geometry}
\usepackage{setspace}
\spacing{1.5}
%\usepackage[version=4]{mhchem}

\author{Carlos Molina Ordóñez}
\date{20 de Marzo de 2017}
\title{Layout de un canal de lectura multi-paralelizado para un sensor de imagen CMOS}

\begin{document}

\maketitle
\tableofcontents

\chapter{Introducción}

\paragraph{}
En este trabajo se va a estudiar todo el proceso de diseño de un bloque fundamental
en cualquier sensor de imagen CMOS, el canal de lectura, que es el encargado
de convertir la información física recibida (número de fotones captados por los
píxeles), a un número digital para posteriormente ser procesado y almacenado digitalmente.

\paragraph{}
El estudio se va a centrar principalmente en el layout de un canal
lectura y en todos los aspectos a tener en cuenta a la hora de abordar esta tarea.
El layout de un sistema microelectrónico consiste en su implementación física
sobre una oblea de algún material semiconductor, habitualmente silicio. El
diseño de layout está sujeto a una serie de normas y problemas que iremos abordando
con mayor detenimiento a lo largo de la exposición.

\paragraph{}
Para introducir al lector en la materia será necesario describir, aunque sea
brevemente, conceptos sobre sensores de imágen, tecnología CMOS y explicar de manera
sencilla la arquitectura de un canal de lectura habitual.

\paragraph{}
Posteriormente se pasará a analizar en detalle los problemas y cuestiones que se
plantean a la hora de diseñar el layout de bloques analógicos en general,
centrandonos en última instancia en los que afectan directamente a un canal de lectura.

\section{Sensores de imagen}

\paragraph{}
Un sensor de imagen o cámara fotográfica es, originalemente, un sistema que capta
una imagen instantánea de una escena mediante la luz que emiten los objetos que
se encuentran en su campo de visión y que llegan a una pantalla donde se almacena
la información que proyecta ese rayo de luz, ya sea por un proceso químico
o electrónico, que es el caso que se va a tratar aquí.

\paragraph{}
En cuanto a los sensores de imágenes electrónicos se pueden distinguir dos tipos
principalmente, los CCD (\textit{Charge-Coupled Device}) y los CMOS
(\textit{Complementary Metal-Oxide-Semiconductor}). Las diferencias entre ellos
se basan en la tecnología empleada y en la forma de leer el array de pixeles.

\subsection{Sensores de imagen CCD}

\paragraph{}
\huge{\textbf{Leer algo en el libro de sensores de imagen!!!!!!!!!}}

\subsection{Sensores de imagen CMOS}

\paragraph{}


\section{Arquitectura de sensores de imagen CMOS}

\section{Array de píxeles}

\paragraph{}
El dispositivo principal en un sensor de imagen es el píxel, que es el elemento
receptor de la luz y el encargado de hacer la primera conversión de fotones a
electrones. Estos fotones se traducen en electrones por medio del efecto fotoeléctrico
descrito por Albert Einstein: Cada fotón que incida en la superficie del silicio
es capáz de dar la energía suficiente a un electrón de la banda de valencia para que
pase a la banda de conducción y se pueda mover libremente por la red cristalina.
Si ahora aplicamos un campo eléctrico podemos guiar a todos los electrones
fotogenerados hacia un pozo de potencial donde almacenarlos temporalmente. Este
es el llamado proceso de exposición, que tiene una duración definida. En función
del tiempo de exposición y de la intensidad lumínica recibida por el pixel, este
se cargará con mayor o menor número de electrones.

\paragraph{}
Observando el esquemático de un pixel 5T (5 transistores) habitual, podemos entender
el funcionamiento del mismo. El fotodiodo (PD) representa la zona donde los fotones
inciden y donde se convierten a electrones durante el tiempo de exposición. Al final
de éste periodo se abre la compuerta (TX), también llamada \textit{transfer gate},
y los electrones pasan por difusión hacia un pozo de potencial llamado \textit{floating
diffusion} (FD), que se representa como un condensador.

\paragraph{}
Posteriormente, la carga almacenada en la \textit{floating diffusion} se convierte
en voltaje a través del seguidor por fuente (SF) cuando la fila actual es seleccionada
para ser leída mediante la activación del transistor de SEL, que carga la columna
de lectura con la tensión a la salida del \textit{source follower}. Este valor de
tensión es el que será leído por el Canal de Lectura.

\paragraph{}
Durante todo el proceso
se pueden producir dos operaciones de limpieza o puesta a cero. Una la hace el
transistor de anti-blooming (BMG), que se encarga de limpiar el diodo de posibles
electrones fotogenerados antes del tiempo de exposición. Este transistor también
realiza la fucnión de "desagüe" en el caso de que la intensidad lumínica sea tal
que el número de electrones en el fotodiodo sature, en cuyo caso, este transistor
drenará la carga sobrante hacia la alimentación. La otra limpieza la
lleva a cabo el transistor de Reset (RST), que reestableze el valor "cero" en la
 \textit{floating diffusion} antes de volcar los nuevos electrones generados en
 la exposición actual.




\paragraph{}
El canal de lectura de un sensor CMOS, referido habitualmente por sus
siglas en inglés \textbf{RO} (\textit{Read-Out Channel}), es el bloque que se
encarga de traducir el voltaje almacenado en cada pixel durante el proceso
de exposición, en un número digital. Esta descripción concuerda con el concepto
ampliamente utilizado en electrónica de ADC, siglas en inglés de \textit{Analog-to-Digital
Converter}, (Convertidor Analógico-Digital), que, en general toma una señal
analógica y la expresa en valores discretizados

\chapter{Tecnología CMOS}

\section{Introducción}

\paragraph{}
La tecnología CMOS, que es ampliamente utilizada en el diseño de circuitos integrados
en la actualidad, se basa en la posibilidad de integrar en un mismo substrato
semiconductores con ambos dopados (N y P). Con ello podemos implementar transistores
MOSFET tanto PMOS como NMOS en un mismo diseño.

\paragraph{}
Es una tecnología que cumple ya los
50 años, debido a que empezó a ponerse en práctica a mediados de los años 1960.
Inicialmente se usó principalmente en circuitos digitales, debido a que los
transistores CMOS solo consumen potencia cuando conmutan, a diferencia de los
transistores de unión bipolar. Por otra parte, es más sencillo disminuir su tamaño
y tienen un menor coste de fabricación.

\paragraph{}
Poco a poco se fue introduciendo la tecnología CMOS en el diseño de circuitos
analógicos. Los bajos costes y la posibilidad de crear circuitos digitales y
analógicos en el mismo chip hacían esta opción muy interesante. Pero aún así, los
transistores bipolares eran mucho menos ruidosos y más rápidos que los MOSFET, por
lo que la trancisión fue lenta. Con el desarrollo de la tecnología CMOS, la velocidad
y el ruido de éstos, se ha visto muy mejorada, y actualmente domina el mercado,
aunque en muchos casos se sigue usando tecnología bipolar.
% Mirar un poco más esto (Historia y actual uso de las tecnologías del silicio)


\section{Proceso de fabricación}

La gran mayoría de los circuitos integrados CMOS estan construidos sobre silicio.
El silicio (Si), elemento de número atómico 14, es muy abundante en la Tierra, aunque
no se encuentra de forma pura, sino como óxidos de silicio o silicatos. Entre los
óxidos de silicio, basados fundamentalmente en la sílice o dióxido de silicio
(SiO\textsubscript{2}), se encuentran el cuarzo y el sílex, ambos ampliamente extendidos
en la corteza terrestre. Los silicatos son sales basadas en el ión silicato (SiO\textsubscript{4}),
y forman parte de minereales como los feldespatos, micas, berilio.

\huge{\textbf{Leer algo en el libro de minerales y rocas en casa!!!!!!!!!}}

A pesar de estar en tan alta abundancia en la Tierra, como se dijo antes, el silicio
puro no se dá naturalmente, debe ser refinado y cristalizado. El proceso consiste,
tratado de manera sencilla, en extraer el óxigeno de los compuestos mencionados arriba
a base de añadir carbono y fundir la mezcla en un horno. Tras éste y otros porcesos
obtendríamos silicio relativamente puro, pero en forma policristalina (habitualmente
nos referiremos a éste como polisilicio). Esta forma contiene silicio puro, pero
cristalizado en pequeños cristales independientes con diferentes planos cristalinos
creando efectos de borde en el interior del conglomerado, que anulan los efectos
semiconductores del silicio.

Para construir un único monocristal de silicio se suele usar el llamado proceso
de Czochralski, en el cual, una varilla de silicio usada como semilla se va
rotando en un baño de silicio puro fundido a unos 1400ºC de manera que va creciendo
en diámetro a medida que los átomos se silicio se van depositando en la capa externa.

Lo que queda es un lingote de aproximadamente un metro de largo y pocas decenas
de centímetro de diámetro de silicio monocristalino siguiendo la esctructura cristalina del
diamante.

Para ser usado en la industria de semiconductores, estos lingotes se deben laminar
en obleas de pocos milímetros de espesor sobre las que se impmlementarán los
circuitos integrados.

\section{Transistores CMOS}

\section{Técnicas avanzadas en tecnología CMOS}

\section{Diseño de layout}

\paragraph{}
El término \textit{layout} hace referencia a la implementación física del circuito
electrónico que se quiere fabricar. El layout consiste en un dibujo con toda
la información que necesita la empresa fabricante para implementar el circuito
sobre una oblea de silicio. Dicha información se representa por medio de "capas"
que se distribuyen en un espacio bidimensional. Cada capa tiene un significado y
unas normas que cumplir. El trabajo del diseñador de layout es definir, dibujar y
verificar el layout de los diferentes bloques que componen el chip siguiendo
unas normas dadas por el fabricante.

\subsection{Capas de layout}

\paragraph{}
Todo diseño de layout está compuesto de una cierta cantidad de capas, de las cuales
algunas tienen significado físico directo y otras son capas que usa el software de
diseño para verificaciones.

\subsection{Herramientas de CAD}

\subsection{Problemas habituales en el diseño de layout}

\chapter{Diseño electrónico de un canal de lectura}

\section{Estructura general}

\section{Arquitectura del comparador de rampa}

\section{ADC}

\section{Rampa analógica}

\section{Fuente de corriente}

\section{Bloques de polarización}

\section{Rampa digital y serialización}

\chapter{Diseño del layout de un canal de lectura}

\section{Introducción}

\section{Jerarquización}

\section{Aproximación al layout de una columna del canal}

\section{Problemas de layout que afectan al diseño del canal de lectura}

\subsection{Acoplos}

\subsection{Concentración de routing}

\subsection{Consumo}

\subsection{Distribución de señales horizontalmente}

\subsection{Condensadores del ADC}

\section{Bloques adicionales}

\subsection{Redundancia}

\subsection{Binning}

\subsection{Implementación de ROIs}

\section{Aproximación a arquitecturas de canales de lectura más complejas}

\subsection{Aumento de la multiparalelización}

\subsection{Sistemas de bajo ruido}

\subsection{Arquitecturas innovadoras}
Modo corriente

\chapter{Conclusiones}

\section{Dificultad creciente de los canales de lectura}

\section{Condicionamiento del diseño del canal por aspectos de layout}

\section{Estudios de viabilidad adecuados}

\section{Uso de nuevas tecnologías}

\section{Innovaciones en las tecnologías de los píxeles}

\chapter{Referencias}

\end{document}
